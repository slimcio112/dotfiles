\documentclass[12pt,a4paper]{article}

% --- PAKIETY PODSTAWOWE ---
\usepackage[utf8]{inputenc}
\usepackage[T1]{fontenc}
\usepackage[polish]{babel}

% --- GEOMETRIA STRONY ---
\usepackage[margin=2.5cm, top=3cm, bottom=3cm]{geometry}

% --- PAKIETY MATEMATYCZNE ---
\usepackage{amsmath, amssymb, amsthm, mathtools}

% --- KOLORY I HIPERŁĄCZA ---
\usepackage{xcolor}
\definecolor{ctpLatteBase}{HTML}{eff1f5}
\definecolor{ctpLatteCrust}{HTML}{dce0e8}
\definecolor{ctpLatteText}{HTML}{4c4f69}
\definecolor{ctpLatteSubtext}{HTML}{6c6f85}
\definecolor{ctpLatteMauve}{HTML}{8839ef}
\definecolor{ctpLatteBlue}{HTML}{1e66f5}
\definecolor{ctpLatteGreen}{HTML}{40a02b}
\definecolor{ctpLatteRed}{HTML}{d20f39}
\definecolor{ctpLattePeach}{HTML}{fe640b}
\definecolor{ctpLatteLavender}{HTML}{7287fd}

\usepackage{hyperref}
\hypersetup{
    colorlinks=true,
    linkcolor=ctpLatteBlue,
    urlcolor=ctpLattePeach,
    citecolor=ctpLatteGreen,
    pdftitle={Notatki z Matematyki},
    pdfauthor={Twoje Imię},
}

% --- TCOLORBOX ---
\usepackage[most]{tcolorbox}

\newtcolorbox{Defi}{
    enhanced,
    colback=ctpLatteBase,
    colframe=ctpLatteMauve,
    coltitle=white,
    fonttitle=\bfseries,
    title=Definicja,
    boxed title style={
        colback=ctpLatteMauve,
        colframe=ctpLatteMauve,
        coltext=white,
        boxrule=1pt,
        arc=4pt
    },
    attach boxed title to top left={xshift=6pt,yshift=-2pt},
    boxrule=1pt,
    arc=4pt,
    left=8pt,right=8pt,top=6pt,bottom=6pt
}

\newtcolorbox{Tw}{
    enhanced,
    colback=ctpLatteBase,
    colframe=ctpLatteBlue,
    coltitle=white,
    fonttitle=\bfseries,
    title=Twierdzenie,
    boxed title style={
        colback=ctpLatteBlue,
        colframe=ctpLatteBlue,
        coltext=white,
        boxrule=1pt,
        arc=4pt
    },
    attach boxed title to top left={xshift=6pt,yshift=-2pt},
    boxrule=1pt,
    arc=4pt,
    left=8pt,right=8pt,top=6pt,bottom=6pt
}

\newtcolorbox{Prz}{
    enhanced,
    colback=ctpLatteBase,
    colframe=ctpLatteGreen,
    coltitle=white,
    fonttitle=\bfseries,
    title=Przykład,
    boxed title style={
        colback=ctpLatteGreen,
        colframe=ctpLatteGreen,
        coltext=white,
        boxrule=1pt,
        arc=4pt
    },
    attach boxed title to top left={xshift=6pt,yshift=-2pt},
    boxrule=1pt,
    arc=4pt,
    left=8pt,right=8pt,top=6pt,bottom=6pt
}

\newtcolorbox{Uw}{
    enhanced,
    colback=ctpLatteBase,
    colframe=ctpLattePeach,
    coltitle=white,
    fonttitle=\bfseries,
    title=Uwaga,
    boxed title style={
        colback=ctpLattePeach,
        colframe=ctpLattePeach,
        coltext=white,
        boxrule=1pt,
        arc=4pt
    },
    attach boxed title to top left={xshift=6pt,yshift=-2pt},
    boxrule=1pt,
    arc=4pt,
    left=8pt,right=8pt,top=6pt,bottom=6pt
}

% --- LEPSZE NAGŁÓWKI SEKCJI ---
\usepackage{titlesec}
\titleformat{\section}
  {\normalfont\Large\bfseries\color{ctpLatteLavender}}
  {\thesection.}{1em}{}
\titleformat{\subsection}
  {\normalfont\large\bfseries\color{ctpLatteLavender}}
  {\thesubsection.}{1em}{}

% --- POZOSTAŁE ---
\renewcommand{\qedsymbol}{\textcolor{ctpLatteText}{$\blacksquare$}}
\title{\Huge\bfseries Notatki z Matematyki\\[1ex]\large\color{ctpLatteSubtext} Przykładowy dokument}
\author{\large Stanisław Kamzol}
\date{\today}

% =============================================
\begin{document}
\color{ctpLatteText}

\maketitle
\hrule height 1.2pt
\vspace{1.5em}
\tableofcontents
\newpage

\section{Analiza Matematyczna}

\begin{Defi}
Niech $f: \mathbb{R} \to \mathbb{R}$. Mówimy, że funkcja $f$ jest \textbf{ciągła} w punkcie $x_0 \in \mathbb{R}$, jeżeli dla dowolnego ciągu argumentów $(x_n)$ zbieżnego do $x_0$, ciąg wartości funkcji $(f(x_n))$ jest zbieżny do $f(x_0)$. Równoważnie:
\[
 \lim_{x \to x_0} f(x) = f(x_0).
\]
\end{Defi}

\begin{Tw}
\textbf{Twierdzenie Weierstrassa:} Każda funkcja ciągła określona na przedziale domkniętym i ograniczonym $[a,b]$ przyjmuje w tym przedziale swoją wartość największą (maksimum) i najmniejszą (minimum).
\end{Tw}

\begin{proof}
Dowód opiera się na pojęciu zwartości. Przedział domknięty $[a,b]$ jest zbiorem zwartym w $\mathbb{R}$. Obraz zbioru zwartego przez funkcję ciągłą również jest zbiorem zwartym. Zbiór zwarty w $\mathbb{R}$ jest domknięty i ograniczony, a zatem zawiera swoje kresy (supremum i infimum).
\end{proof}

\begin{Prz}
\textbf{Funkcja sinus:} Funkcja $f(x)=\sin(x)$ jest ciągła na całej prostej rzeczywistej. Na dowolnym przedziale domkniętym, np. $[0, 2\pi]$, osiąga swoje kresy:
\begin{itemize}
    \item Wartość maksymalną: $1$, dla $x = \frac{\pi}{2}$.
    \item Wartość minimalną: $-1$, dla $x = \frac{3\pi}{2}$.
\end{itemize}
Jej zbiór wartości to $[-1,1]$.
\end{Prz}

\begin{Uw}
\textbf{Uwagi dotyczące przedziałów:} Założenie o domkniętości przedziału jest kluczowe. Na przykład funkcja $g(x) = \frac{1}{x}$ jest ciągła na przedziale otwartym $(0,1]$, ale nie jest na nim ograniczona z góry, więc nie przyjmuje maksimum.
\end{Uw}

\end{document}

